%
% $Id: PPI.tex,v 1.4 2004/03/21 14:07:00 nordstrom Exp $
%
\chapter{Plucker Plugin Interface}\label{sec:PPI}

The Plucker Plugin Interface (PPI) is a separate program (e.g.,
\code{ppi\_en.prc}) used to interface Plucker with external programs
like dictionaries.  To use PPI to look up words, configure the Viewer
as described in \~ref{sec:Lookup}.\\

PPI has hard-coded settings for the popular dictionary programs RoadLingua,
KDic DA, and BDicty.  To use with other dictionaries or encyclopedias, ask
their customer support for appropriate data to enter into the \option{Custom}
setting.\\

The RoadLingua dictionary viewer is commercial, but the developer has made the
free (as in beer) unregistered version act like the registered version 
(at least with non-commercial dictionary texts) when it is invoked via PPI.  
You may need the latest version of RoadLingua for this from
\textit{\htmladdnormallink{http://www.absoluteword.com}
{http://www.absoluteword.com}}.\\

Note that when using the free version of BDicty, the word will
be put into the clipboard and will need to be pasted into BDicty.\\

PPI is available for use with other programs than Plucker.  For instance,
currently PalmBible+ uses it for its dictionary lookup;  see:
\textit{\htmladdnormallink{http://palmbibleplus.sf.net}
{http://palmbibleplus.sf.net}}.

