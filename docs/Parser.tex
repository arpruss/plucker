%
% $Id: Parser.tex,v 1.15 2003/07/31 15:12:06 desrod Exp $
%
\chapter{The Parser}\label{sec:Parser}

As pointed out in the previous chapter the parser is more or less the
workhorse of Plucker. This implies that most customizations are done to
the parser and that the parser covers a rather big part of this manual.

\section{Description file}\label{sec:Home Page}

Before we take a closer look at the parser itself, we will describe
the format of the description file also known as the Home Page document
(default \name{home.html}, but that can be changed). On a Unix/Linux
system this file will be stored by default in \name{\$HOME/.plucker}.

OS/2 will use the environment-variable \textit{HOME} to find the
location of your home-directory (you can also use drive letters).
The installer should set the necessary environment variable for
you and also add the necessary directories to your system. You
may check the location by simply typing \code{set home} at a
command prompt.\\

The description file is a valid HTML document with extra optional
tags added for the link references.

\begin{itemize}
 \item \longcode{description file}{MAXDEPTH=n}: This specifies how
 deep the parser should follow the links embedded in a web page.
 If \code{MAXDEPTH} is not given the parser will default to a depth
 of 1, that is only download the page itself but do not follow any
 links in it. To follow links in the current page you would use
 \code{MAXDEPTH=2} and to follow links also in those pages you would
 use \code{MAXDEPTH=3} and so on. Too high values without using any
 of the available filtering mechanisms could result in an excessive
 amount of data.

 \hint{} \code{MAXDEPTH=2} can be very useful if you have a page that
 contains only the headlines that are links to the full text version
 of the articles. Many newstickers use this format.

 \item \longcode{description file}{NOIMAGES}: If you are not interested
 in downloading images
 then you use this tag. If specified all images will be replaced with
 the ALT-tag for the image if available, otherwise \textbf{[img]}.

 \hint{} \code{NOIMAGES} is an effective way to decrease the size of
 documents.

 \item \longcode{description file}{STAYONHOST}: Most web sites contains
 references to both locally stored articles and to articles stored on
 other hosts. Using a \code{MAXDEPTH} of 2 or higher could result in a
 lot of unwanted data.  To prevent this you may specify the \code{STAYONHOST}
 tag for your link. The parser will now only download content that resides
 on the same server as the one that contained the top page. Together with
 \iname{exclusionlist.txt} this is a quite handy way to prevent the download
 of links referred to by banners.

 \item \longcode{description file}{STAYBELOW=\emph{text}}: Similar to
 \code{STAYONHOST} this tag tells the parser to only fetch pages that
 start with \emph{text}.  For example, it could be used if the
 articles on a page are listed on another server in which case
 \code{STAYONHOST} would not work properly.  Or you can grab certain
 articles out of a large listing so you would get all headlines but
 only articles referring to specific subjects (provided the web server
 offering the information is set up \emph{correctly}).

 \note If \emph{=text} is not given, it will default to the content
 of the href-attribute (the URL you are pointing to).

 \item \longcode{description file}{BPP=n}: This option is used to specify
 the bit depth that should be used for images. Valid values are 0 (i.e.\
 no images), 1, 2, 4, and 8.

 \note{} \code{BPP=8} is currently only supported when the parser is used
 on a Windows system.

 \item \longcode{description file}{MAXWIDTH=width}: Used to set the maximum
 width of images.

 \item \longcode{description file}{MAXHEIGHT=height}: Used to set the maximum
 height of images.

\end{itemize}

An simple example of a description file is:

\begin{verbatim}
<HTML>
<HEAD>
  <TITLE>Plucker Home Page</TITLE>
</HEAD>
<BODY>
  <A HREF="http://www.plkr.org" MAXDEPTH=2 STAYONHOST NOIMAGES>Plucker home page</A>
</BODY>
</HTML>
\end{verbatim}

This would download the front page of our web site and also follow any
links on the page if they are local to the host. No images would be
downloaded.\\

The description file (\name{home.html}) that is installed when your
Plucker directory is set up, also contains a few examples.\\

\section{Invoking the parser}

As the previous section showed, you can specify several options specific
for each reference in the description file and many times this is enough.
But additionally, the parser provides some global control through
powerful command-line options and also a possibility to use different
configurations that have been set up in the config file.\\

In all the examples below we will call \name{Spider.py} directly. This
could be stored in a script or we could create a symbolic link to it,
e.g.\ if you have installed the Linux package you would use \name{plucker-build}
instead of \name{Spider.py} in the examples. A note for OS/2 and Windows
users: running \name{Spider.py} as an executable program is a Unix trick
and may not work on your platform. But do not worry, just list the
\name{Spider.py} on an explicit \name{python} command line, that is use
\code{python Spider.py \ldots} instead. 

The synopsis of the parser look like this:\\

\begin{latexonly}
\begin{tabular}{lp{15cm}}
  \ioption{Spider.py} & \option{\mbox{-c}\textbar \mbox{-{}-update-cache}\textbar 
  \mbox{-f FILE}\textbar \mbox{-{}-doc-file=FILE} 
  [\mbox{-N NAME}\textbar \mbox{-{}-doc-name=NAME}]
  [\mbox{-h}\textbar \mbox{-{}-help}]} \\
 & \option{[\mbox{-{}-compression=TYPE}\textbar \mbox{-{}-zlib-compression}\textbar k
  \mbox{-{}-doc-compression}] [\mbox{-P DIR}\textbar \mbox{-{}-pluckerhome=DIR}]}\\
 & \option{[\mbox{-v}\textbar \mbox{-V NUM}\textbar \mbox{-{}-verbosity=NUM}] 
  [\mbox{-q}\textbar \mbox{-{}-quiet}] [\mbox{-{}-bpp=NUM}] [\mbox{-{}-noimages}] 
  [\mbox{-p DIR}\textbar \mbox{-{}-pluckerdir=DIR}]}\\
 & \option{[\mbox{-H URL}\textbar \mbox{-{}-home-url=URL}] 
  [\mbox{-M DEPTH}\textbar \mbox{-{}-maxdepth=DEPTH}] 
  [\mbox{-E FILE}\textbar \mbox{-{}-exclusion-list=FILE}]}\\
 & \option{[\mbox{-s SECTION}\textbar \mbox{-{}-extra-section=SECTION}]
  [\mbox{-{}-no-urlinfo}] [\mbox{-{}-stayonhost}] [\mbox{-{}-staybelow=URLPREFIX}] [\mbox{-{}-category=CATEGORY}]}
\end{tabular}\\
\end{latexonly}
\begin{htmlonly}
\index{Spider.py}
\end{htmlonly}
\begin{rawhtml}
  <B>Spider.py -c | --update-cache | -f FILE | --doc-file=FILE [-N NAME | --doc-name=NAME]
  [-h | --help] [--compression=TYPE | --zlib-compression | --doc-compression] 
  [-P DIR | --pluckerhome=DIR] [-v | -V NUM | --verbosity=NUM] [-q | --quiet] 
  [--bpp=NUM] [--noimages] [-p DIR | --pluckerdir=DIR] [-H URL | --home-url=URL] 
  [-M DEPTH | --maxdepth=DEPTH] [-E FILE | --exclusion-list=FILE] 
  [-s SECTION | --extra-section=SECTION] [--no-urlinfo] [--stayonhost]
  [--staybelow=URLPREFIX] [--category=CATEGORY]</B>
\end{rawhtml}

Do not run away in fear of all these options, you don't have to use
all of them\ldots actually, you only have to use \textbf{one} of them
(and that can be reduced to none if you set the file name of the
document in the configuration file --- more about that later).\\

\begin{latexonly}
You can always get the list of the parameters by using the \longoption{Spider.py}{-h}
or \longoption{Spider.py}{-{}-help} option.
\end{latexonly}
\begin{htmlonly}
You can always get the list of the parameters by using the \longoption{Spider.py}{-h}
or \longoption{Spider.py}{---help} option.
\end{htmlonly}

\subsection{Control Over the Result}\label{sec:ParserOutput}

\begin{latexonly}
The \longoption{Spider.py}{-c}, \longoption{Spider.py}{-{}-update-cache}, 
\longoption{Spider.py}{-f}, and \longoption{Spider.py}{-{}-doc-file}
options allows you to specify if the parser should dump all records in
a cache directory or if it shall create a document. For example, the command,
\end{latexonly}
\begin{htmlonly}
The \longoption{Spider.py}{-c}, \longoption{Spider.py}{---update-cache}, 
\longoption{Spider.py}{-f}, and \longoption{Spider.py}{---doc-file}
options allows you to specify if the parser should dump all records in
a cache directory or if it shall create a document. For example, the command,
\end{htmlonly}\\

\indata{\% Spider.py -c}\\

\begin{latexonly}
tells the parser to dump all records in the cache directory in \code{\$PLUCKERDIR}
using the default description file (as you will see below it is possible
to change the default description file in Plucker's config file). The 
\option{-{}-update-cache} option works the same way as \option{-c}.
\end{latexonly}
\begin{htmlonly}
tells the parser to dump all records in the cache directory in \code{\$PLUCKERDIR}
using the default description file (as you will see below it is possible
to change the default description file in Plucker's config file). The 
\option{---update-cache} option works the same way as \option{-c}.
\end{htmlonly}\\

If we instead used the following command,\\

\indata{\% Spider.py -f PluckerDB}\\

\begin{latexonly}
the parser would create a document called \name{PluckerDB.pdb} and store
it in \code{\$PLUCKERDIR}. The \option{-{}-doc-file} option works the same
way as \option{-f}. Either \option{-c} or \option{-f} must be given unless
the document has been specified in the configuration file (see
\~ref{sec:ConfigFile}).  The name of the document will default to the same
as the name given to \option{-f}. If you want to give the document a
different name, you can use the \longoption{Spider.py}{-N} or
\longoption{Spider.py}{-{}-doc-name} option,
\end{latexonly}
\begin{htmlonly}
the parser would create a document called \name{PluckerDB.pdb} and store
it in \code{\$PLUCKERDIR}. The \option{---doc-file} option works the same
way as \option{-f}. Either \option{-c} or \option{-f} must be given unless
the document has been specified in the configuration file (see
\~ref{sec:ConfigFile}). The name of the document will default to the same
as the name given to \option{-f}. If you want to give the document a
different name, you can use the \longoption{Spider.py}{-N} or
\longoption{Spider.py}{---doc-name} option,
\end{htmlonly}\\

\indata{\% Spider.py -f PluckerDB -N "Home Page"}\\

This will still create a document called \name{PluckerDB.pdb} in
\code{\$PLUCKERDIR}, but the name of the document (i.e.\ the name you
would see on your Palm) would now be \name{Home Page}.\\

The file name given to the \option{-f} option can contain an absolute
or relative path, so if we run,\\

\indata{\% Spider.py -f /home/pilot/.plucker/DB/PluckerDB}\\

a document called \name{PluckerDB.pdb} will be stored in
\name{/home/pilot/.plucker/DB} and if we run,\\

\indata{\% Spider.py -f DB/PluckerDB}\\

a document called \name{PluckerDB.pdb} will be stored in
\name{\$PLUCKERDIR/DB} (the same as above since \code{\$PLUCKERDIR}
in this case defaults to \name{/home/pilot/.plucker/})\\

To save space on the device all records are compressed and the
parser provides two different types of compression, DOC compression
and ZLib compression. The ZLib compression gives the best compression,
but is only available for Palm OS 3.0 and later devices and also
requires a shared library.\\

\begin{latexonly}
You control the compression type by using the
\longoption{Spider.py}{-{}-compression=TYPE} option. 
Allowable options are `doc', for Palm DOC compression, 
or `zlib', for zlib compression. You can also use the
\longoption{Spider.py}{-{}-zlib-compression} or the
\longoption{Spider.py}{-{}-doc-compression} option. If
neither is given the parser defaults to DOC compression.
\end{latexonly}
\begin{htmlonly}
You control the compression type by using the
\longoption{Spider.py}{---compression=TYPE} option. 
Allowable options are `doc', for Palm DOC compression, 
or `zlib', for zlib compression. You can also use the
\longoption{Spider.py}{---zlib-compression} or the
\longoption{Spider.py}{---doc-compression} option. If
neither is given the parser defaults to DOC compression.
\end{htmlonly}\\

\begin{latexonly}
To put the document in a specific category by default you can use
the \longoption{Spider.py}{-{}-category} option. It is possible
to give several default categories separated by ";".
\end{latexonly}
\begin{htmlonly}
To put the document in a specific category by default you can use
the \longoption{Spider.py}{---category} option. It is possible 
to give several default categories separated by ";".
\end{htmlonly}

\subsection{Status Information and Debugging}

\begin{latexonly}
The parser normally outputs errors and progress status to your terminal,
but by using the \longoption{Spider.py}{-q}, \longoption{Spider.py}{-{}-quiet},
\longoption{Spider.py}{-v}, \longoption{Spider.py}{-V} or
\longoption{Spider.py}{-{}-verbosity} option you can change this. The \option{-V}
(and \option{-{}-verbosity}) takes an argument,
\end{latexonly}
\begin{htmlonly}
The parser normally outputs errors and progress status to your terminal,
but by using the \longoption{Spider.py}{-q}, \longoption{Spider.py}{---quiet},
\longoption{Spider.py}{-v}, \longoption{Spider.py}{-V} or
\longoption{Spider.py}{---verbosity} option you can change this. The \option{-V}
(and \option{---verbosity}) takes an argument,
\end{htmlonly}

\begin{enumerate}
\item[0] silent, except for errors.
\item[1] progress status (default), will list the page currently being plucked
as well as some output from the image converters.
\item[2] produces a lot of output about the gathering and parsing process,
used for debugging.
\end{enumerate}

\begin{latexonly}
The \option{-v} option sets verbosity level 1, while \option{-q} and
\option{-{}-quiet} set verbosity level 0.
\end{latexonly}
\begin{htmlonly}
The \option{-v} option sets verbosity level 1, while \option{-q} and
\option{---quiet} set verbosity level 0.
\end{htmlonly}

\subsection{Description filenames}

\begin{latexonly}
The \longoption{Spider.py}{-H} or \longoption{Spider.py}{-{}-home-url} 
option allows you to specify a different description file than the 
default one. For example, the command,
\end{latexonly}
\begin{htmlonly}
The \longoption{Spider.py}{-H} or \longoption{Spider.py}{---home-url} 
option allows you to specify a different description file than the 
default one. For example, the command,
\end{htmlonly}\\

\indata{\% Spider.py -H plucker:/test.html -f TestDB}\\

tells the parser to build a document called \name{TestDB.pdb} using the
description file \name{test.html} that can be found in \code{\$PLUCKERDIR}
(\name{plucker:} is relative to \code{\$PLUCKERDIR}). You can also use
an absolute path,\\

\indata{\% Spider.py -H file:/home/pilot/.plucker/test.html -f TestDB}\\

would produce the same result as the first example.\\

Instead of using a local file we can use a file at a remote location,\\

\indata{\% Spider.py -H http://www.plkr.org/ -f PluckerDB}\\

and the parser would use the index page for Plucker's home page as the
description file.\\

\note{} The \longoption{Spider.py}{-H} option needs a valid URL, so
don't use, \\

\indata{\% Spider.py -H file:test.html -f TestDB}\\

even if you are running the spider in the same directory as the file is
located in.

\subsection{How deep should we go?}\label{sec:Depth}

\begin{latexonly}
The default operation of the parser is to parse the description file
and any links in it, but this can be changed using the \longoption{Spider.py}{-M}
or \longoption{Spider.py}{-{}-maxdepth} option. To make sure that we
do not bite off more than we can chew, we can add the
\longoption{Spider.py}{-{}-stayonhost} option to avoid following any
external links,\\

\indata{\% Spider.py -H http://www.plkr.org/ -M3 -{}-stayonhost -f PluckerDB}
\end{latexonly}
\begin{htmlonly}
The default operation of the parser is to parse the description file
and any links in it, but this can be changed using the \longoption{Spider.py}{-M}
or \longoption{Spider.py}{---maxdepth} option. To make sure that we
don't bite off more than we can chew, we can add the \longoption{Spider.py}{---stayonhost}
option to avoid following any external links,\\

\indata{\% Spider.py -H http://www.plkr.org/ -M3 ---stayonhost -f PluckerDB}
\end{htmlonly}\\

This would download the index page for Plucker's web site, any links
within this page and also any links within those pages, but it would
disregard any links outside the \name{plkr.org} host. \\

These options are useful when you specify a specific host using
-H. If you use a description file it is better to use the
MAXDEPTH/STAYONHOST flags for each link instead.

\subsection{Plucker Home and the Plucker Directory}

The Plucker directory (\code{\$PLUCKERDIR}) is the current working
directory, while Plucker Home is like your home directory. They can
be the same directory and the default behavior is that the Plucker
directory is set to the same as Plucker Home (default for Plucker
Home is \name{\$HOME/.plucker}, but you can change this by using
the environment variable \code{\$PLUCKERHOME}). \name{plucker:/XXX}
files are first searched in \code{\$PLUCKERDIR} then Plucker Home
and any configuration files in the \code{\$PLUCKERDIR} take precedence
over those in the Plucker Home.\\

\begin{latexonly}
The location of Plucker Home can be changed using the \longoption{Spider.py}{-P}
or \longoption{Spider.py}{-{}-pluckerhome} option and the location of
the Plucker directory can be changed using the \longoption{Spider.py}{-p}
or \longoption{Spider.py}{-{}-pluckerdir} option,
\end{latexonly}
\begin{htmlonly}
The location of Plucker Home can be changed using the \longoption{Spider.py}{-P}
or \longoption{Spider.py}{---pluckerhome} option and the location of
the Plucker directory can be changed using the \longoption{Spider.py}{-p}
or \longoption{Spider.py}{---pluckerdir} option,
\end{htmlonly}\\

\indata{\% Spider.py -P \. -H plucker:/test.html -f TestDB}\\

would use the current directory as Plucker Home and the file \name{test.html}
should also be in this directory.


\subsection{Image handling}

\begin{latexonly}
The \longoption{Spider.py}{-{}-bpp} and \longoption{Spider.py}{-{}-noimages}
options can be used to globally set the image handling. The \option{-{}-bpp}
option takes an argument for the bits per pixels that should be used,
0, 1 (default), 2, 4, or 8.  The \option{-{}-noimages} option sets the
bit depth to 0. To create images in 16 colors (grayscale) you would use,
\end{latexonly}
\begin{htmlonly}
The \longoption{Spider.py}{---bpp} and \longoption{Spider.py}{---noimages}
options can be used to globally set the image handling. The \option{---bpp}
option takes an argument for the bits per pixels that should be used,
0, 1 (default), 2, 4, or 8.  The \option{---noimages} option sets the
bit depth to 0. To include images in 16 colors (grayscale) you would use,
\end{htmlonly}\\

\indata{\% Spider.py ---bpp=4 -H plucker:/images.html -f Gray16DB}


\subsection{URL handling}

\begin{latexonly}
To save space on you device you can choose to not include information
about the URLs. This will mean that you cannot get access to the actual
URL for documents in the document or for any external reference. Use
the \longoption{Spider.py}{-{}-no-urlinfo} option if you don't want to
include this information.
\end{latexonly}
\begin{htmlonly}
To save space on you device you can choose to not include information
about the URLs. This will mean that you cannot get access to the actual
URL for documents in the document or for any external reference. Use
the \longoption{Spider.py}{---no-urlinfo} option if you don't want to
include this information.
\end{htmlonly}


\subsection{Excluding Things}

\begin{latexonly}
To add more exclusion lists than the default ones you can use the
\longoption{Spider.py}{-E} or \longoption{Spider.py}{-{}-exclusion-list}
option. The \name{exclusionlist.txt} files should be in a specific
format, that is explained in section \~ref{sec:Exclusionlist}. You
can add more than one \option{-E},
\end{latexonly}
\begin{htmlonly}
To add more exclusion lists than the default ones you can use the
\longoption{Spider.py}{-E} or \longoption{Spider.py}{---exclusion-list}
option. The \name{exclusionlist.txt} files should be in a specific
format, that is explained in section \~ref{sec:Exclusionlist}. You
can add more than one \option{-E},
\end{htmlonly}\\

\indata{\% Spider.py -H plucker:/exclude.html -E Exclude/default.excl -E Exclude/extra.excl -f ExcludeDB}\\

would use \name{default.excl} and \name{extra.excl} found in
\name{\$PLUCKERDIR/Exclude} as extra exclusion lists.


\subsection{Sections}

\begin{latexonly}
The \longoption{Spider.py}{-s} or \longoption{Spider.py}{-{}-extra-section}
option is used to add sections to the list of searched sections in the
config file(s). You can read more about sections in \~ref{sec:ConfigFile}
and \~ref{sec:Examples}.
\end{latexonly}
\begin{htmlonly}
The \longoption{Spider.py}{-s} or \longoption{Spider.py}{---extra-section}
option is used to add sections to the list of searched sections in the
config file(s). You can read more about sections in \~ref{sec:ConfigFile}
and \~ref{sec:Examples}.
\end{htmlonly}


\section{The Configuration File}\label{sec:ConfigFile}

The configuration file allows you to specify default values for global
options. Under OS/2 and Windows this file should be called \name{plucker.ini}.
For Windows, put this in Plucker Home or in the Plucker directory
(\code{\$PLUCKERDIR}). Under Unix/Linux, this file should be called
\name{pluckerrc} if it is the system wide config file and \name{.pluckerrc}
if it is a user config file. It should have been installed in the correct
place (e.g. \name{.pluckerrc} should be in your home directory.)\\

Entries are \code{key = value} pairs ordered into sections. A section
is named and begins with a line in square bracket, where the brackets
contain the section's name. Lines starting with a \emph{;;} are considered
comments.\\

All systems search the \longoption{configuration file}{[DEFAULT]} section.
Under Windows, the \longoption{configuration file}{[WINDOWS]} section
is also searched, OS/2 will search both the \longoption{configuration file}{[OS2]}
and \longoption{configuration file}{[POSIX]} section and all other systems
will search the \option{[POSIX]} section. Using the \longoption{Spider.py}{-s}
on the command line you can add other sections to search. The parser starts
with the \option{[DEFAULT]} section, then the system dependent section(s)
and last any sections defined on the command line. If a key is repeated
the value for the last one will be used.\\

The following \code{keys} are defined (check the config file that came with
your version of Plucker, new keys might have been added after this version
of the document was released):

\subsection{[DEFAULT]}

\begin{description}
\item[verbosity] verbosity level (integer, default: 1).
\item[bpp] default bits per pixel (integer, default: 1)
\item[db\_name] name (on the Palm) of the DB (string, no default)
\item[db\_file] filename for the db relative to the plucker dir
(string, no default)
\item[cache\_dir\_name] name of the directory where cache files will
be stored (string, default: cache)
\item[home\_url] the URL to home document (string, default:
plucker:/home.html)
\item[home\_maxdepth] max depth to spider the home document
(integer, default: 2)
\item[home\_stayonhost] use STAYONHOST for home\_url (boolean,
default: false)
\item[home\_staybelow] the url part (as STAYBELOW) for home\_url
(string, no default)
\item[maxwidth] max width for the images (string, default: 150)
\item[maxheight] max height for the images (string, default: 250)
\item[pluckerdir] path to the plucker dir (string, default:
\$HOME/.plucker/)
\item[image\_parser] specify which parser to use for converting
images, possible values are \emph{windows}, \emph{netpbm2}, and \emph{imagemagick}.
(defaults to whatever the system determines to be the default parser.)
\item[before\_command] specify commands to be executed before spidering
(string, no default)
\item[after\_command] specify commands to be executed after spidering
(string, no default)
\item[user] hotsync user name (string, no default) (Only used on Windows)
\item[copyprevention\_bit] set the copy prevention bit in the DB.
(boolean, default: false)
\item[backup\_bit] set the backup bit in the DB. (boolean, default: false)
\item[launchable\_bit] set the launchable bit in the DB. (boolean,
default: false)
\item[icon] store an icon in the DBs appinfo block. (boolean, default: false)
\item[small\_icon] path to Tbmp (15x9), if the icon key is set, use
this bitmap for the small icon. (string, default: standard PluckerDB icon)
\item[big\_icon] path to Tbmp (32x32), if the icon key is set, use
this bitmap for the large icon. (string, default: standard PluckerDB icon)
\item[zlib\_compression] a flag indicating whether to use new zlib
compression instead of the doc compression. (boolean, default: false)
\item[no\_image\_compression\_limit] if a Tbmp is smaller than the given
size then it will not be compressed (integer, default 0)
\item[category] default category for the created document, if several 
categories should be used then separate them with ";" (string, default: Unfiled)

\end{description}


\subsection{[POSIX]}

\begin{description}
\item[ppmquant\_program] name (and maybe path) to the ppmquant program
(string, default: `ppmquant')
\item[ppmtoTbmp\_program] name (and maybe path) to the ppmtoTbmp program
(string, default: '\mbox{ppmtoTbmp}')
\item[pnmscale\_program] name (and maybe path) to the pnmscale program
(string, default: `pnmscale')
\item[pnmfile\_program] name (and maybe path) to the pnmfile program
(string, default: `pnmfile')
\item[giftopnm\_program] name (and maybe path) to the giftopnm program
(string, default: `giftopnm')
\item[djpeg\_program] name (and maybe path) to the djpeg program
(string, default: `djpeg')
\item[pngtopnm\_program] name (and maybe path) to the pngtopnm program
(string, default: `pngtopnm')
\item[convert\_program] name (and maybe path) to the convert program
(string, default: `convert')
\item[exclusion\_lists] a list of filenames specifying exclusion lists to 
be inspected.  Names are separated by colons. (default: nothing)
\end{description}


\subsection{[OS2]}

\begin{description}
\item[exclusion\_lists] a list of filenames specifying exclusion lists to 
be inspected.  Names are separated by semicolons. (default: nothing)

\note OS/2 will also evaluate the \longoption{configuration
file}{[POSIX]}. Do not add any exclusion lists there or you may run
into trouble.

\end{description}


\subsection{[WINDOWS]}

\begin{description}
\item[exclusion\_lists] a list of filenames specifying exclusion lists to 
be inspected.  Names are separated by semicolons. (default: nothing)
\item[convert\_program] path and name for the ImageMagick convert tool
with arguments\\
 (default: \mbox{convert.exe \%type\%:\%input\% bmp:\%output\%})
\item[bmp\_to\_tbmp] path and name for the Bmp2Tbmp tool
(default: \mbox{Bmp2Tbmp.exe -i=\%input\% -o=\%output\%}
 \mbox{-maxwidth=\%maxwidth\% -maxheight=\%maxheight\% -compress=\%compress\% -color=\%colors\%})
\item[tbmp\_compression\_type] use compression (string, default: yes)
\item[python\_program] path and name for the Python executable
(string, no default)
\item[group\_path] path of the Plucker program group (string, no default)
\item[http\_proxy] proxy settings of the form, http\_proxy=http://proxy:port
(string, no default)
\item[use\_conduit] use conduit.exe to build PDB (boolean, default: true)
(only for debugging)
\item[close\_on\_exit] close the terminal window when PyPlucker is finished.
(boolean, default: false)
\end{description}

The \option{convert\_program} and \option{bmp\_to\_tbmp} options can be
given the following arguments:
\begin{itemize}
\item \%compress\%  = if compression is used it is replaced with the
string given by the `tbmp\_compression\_type' key, otherwise `no'
\item \%colors\%    = `M', `G4', `G16' or `C256'
\item \%maxwidth\%  = maxwidth value
\item \%maxheight\% = maxheight value
\item \%input\%     = the input filename
\item \%output\%    = the output filename
\item \%type\%      = `JPG', `GIF' or `PNG'
\end{itemize}

\section{The Exclusion List}\label{sec:Exclusionlist}

Sometimes, the \longcode{description file}{STAYONHOST} and
\longcode{description file}{STAYBELOW} attributes do not give enough
selection ability to include only those things that you want included.
The parser looks for so called \name{exclusion lists} in form of text
files. It first loads the file \name{\$HOME/.plucker/exclusionlist.txt}
(if it exists), then it loads \iname{exclusionlist.txt} in the directory
you specified with the \longoption{Spider.py}{-p} flag (again, only if
it exists) and finally it loads any files specified on the command line
using the \longoption{Spider.py}{-E} option and/or found in the
configuration file. A default \name{exclusionlist.txt} file is included
with Plucker.\\

The filter options in the exclusion list has a very flexible format:\\

\begin{latexonly}
\code{\textless prio\textgreater:\textless action\textgreater:\textless regexp\textgreater}
\end{latexonly}
\begin{htmlonly}
\code{<prio>:<action>:<regexp>}
\end{htmlonly}\\

where:
\begin{description}
\begin{latexonly}
  \item[\longcode{exclusion list}{\textless prio\textgreater}] is an
          integer (negative numbers are also valid) specifying the
          priority. Rules with higher priorities are considered before
          rules with lower priorities.  Rules of equal priority are
          considered in the sequence that they appear in the file.
  \item[\longcode{exclusion list}{\textless action\textgreater}] is
          either a plus or a minus sign.  Plus means `include this
          document' while minus means `do not include this document'.
  \item[\longcode{exclusion list}{\textless regexp\textgreater}] a
          valid regular expression (as known, e.g.\ from perl).
\end{latexonly}
\begin{htmlonly}
  \item[\longcode{exclusion list}{<prio>}] is an integer (negative
          numbers are also valid) specifying the priority. Rules with
          higher priorities are considered before rules with lower
          priorities.  Rules of equal priority are considered in
          the sequence that they appear in the file.
  \item[\longcode{exclusion list}{<action>}] is either a plus or a
          minus sign.  Plus means `include this document' while minus
          means `do not include this document'.
  \item[\longcode{exclusion list}{<regexp>}] a valid regular expression
          (as known, e.g.\ from perl).
\end{htmlonly}
\end{description}

Leading and trailing white space is ignored and empty lines or lines
starting with a '\#' are considered comments.\\

Examples:

\begin{verbatim}
0:-:.*\.mp3$
0:-:.*\.wav$

## Ignore known advertisement sites:
0:-:http://.*ad\.de\.doubleclick\.net/.*
0:-:http://.*ad\.linkexchange\.com/.*
\end{verbatim}


\section{Use Cases}\label{sec:Examples}

To give you a few examples on how to use the parser in real life
situations we will describe a number of different use cases. We
start with showing how you \emph{pluck} files located on your own
system and then continues with examples for how to \emph{pluck}
files on remote hosts.\\

In all the examples we will use \name{/home/pilot/.plucker/} as
our Plucker Home (and Plucker Directory). In this directory we
have two subdirectories, one called \name{HTML} where we store
all our different description files and one called \name{DB}
used for storing the resulting documents.

\subsection{Pluck local targets}

The parser can handle any HTML or text document you have on your
desktop system, whether it is simple text files or documents from
a local web server you run on your desktop.

\subsubsection{Creating an E-book}

Handling E-books on your Palm is one of the things that Plucker does
well. To create such E-books you first have to get the book in either
text or (preferable) in HTML format --- \name{Project Gutenberg:
\htmladdnormallink{http://www.promo.net/pg/}{http://www.promo.net/pg/}}
is a good place to find several old classics and other \emph{free} books.
Some books can be found in the \name{Open E-book} (OEB) format. That
format is close enough to HTML to be usable by the parser.\\

In this example we will convert Lewis Carroll's \emph{Alice's Adventures
in Wonderland} using a copy in OEB format that we got from
\name{\htmladdnormallink{http://www.jeffkirvin.com/writingonyourpalm/recommends.htm}
{http://www.jeffkirvin.com/writingonyourpalm/recommends.htm}}. After
unpacking the file in the \name{HTML} subdirectory we have one large
OEB file called \name{alices\_adventures\_in\_wonderland.htm} and the
procedure to convert this file into a Plucker document is very simple,\\

\begin{latexonly}
\indata{\% Spider.py -v -{}-no-urlinfo -H plucker:/HTML/alices\_adventures\_in\_wonderland.htm \textbackslash}\\
\indata{\textgreater{} -N "Alice in Wonderland" -f DB/Wonderland}
\end{latexonly}
\begin{htmlonly}
\indata{\% Spider.py -v ---no-urlinfo -H plucker:/HTML/alices\_adventures\_in\_wonderland.htm \textbackslash}\\
\indata{> -N "Alice in Wonderland" -f DB/Wonderland}
\end{htmlonly}\\
\begin{verbatim}
Working for pluckerdir /home/pilot/.plucker
Processing plucker:/HTML/alices_adventures_in_wonderland.htm.
           0 collected, 0 still to do
  Retrieved ok

Writing out collected data...
Writing db 'Alice in Wonderland' to file /home/pilot/.plucker/DB/Wonderland.pdb
Converted plucker:/HTML/alices_adventures_in_wonderland.htm
Converted plucker:/~parts~/plucker%3a%2fHTML%2fali.....onderland.htm/1
Converted plucker:/~parts~/plucker%3a%2fHTML%2fali.....onderland.htm/2
Converted plucker:/~parts~/plucker%3a%2fHTML%2fali.....onderland.htm/3
Converted plucker:/~parts~/plucker%3a%2fHTML%2fali.....onderland.htm/4
Wrote 1 <= plucker:/~special~/index
Wrote 2 <= plucker:/HTML/alices_adventures_in_wonderland.htm
Wrote 11 <= plucker:/~parts~/plucker%3a%2fHTML%2fali.....onderland.htm/1
Wrote 12 <= plucker:/~parts~/plucker%3a%2fHTML%2fali.....onderland.htm/2
Wrote 13 <= plucker:/~parts~/plucker%3a%2fHTML%2fali.....onderland.htm/3
Wrote 14 <= plucker:/~parts~/plucker%3a%2fHTML%2fali.....onderland.htm/4
Done!

\end{verbatim}

We give it a different name than the document itself and also exclude
the URL info (we don't need that in an E-book). From the output above
you can also see that the document is split into several parts, since
we for internal reasons must keep the text documents below 32 kB in size.\\

The document can be found in the \name{DB} directory,

\subsubsection{An Admin Guide}

The previous example was only using one single file and this example
will show you that it is just as simple when the book is divided up
into several documents. We will use \name{The Linux System Administrators'
Guide} by Lars Wirzenius and Joanna Oja as our test object.\\

Unpacking the files in a separate directory (we will use \name{/tmp/sag/}
for this example) we will find several HTML documents and also a bunch
of GIF images. We are interested in the images to find out what bit
depth we have to use and also their size so we know if they will be
scaled down. In this case all images are black and white, so we can
use the default bit depth of 1. A few of the images are quite large,
so they will be scaled down to 150x250 by the parser. If we really
want these images in full size we can either change the HTML document
so that instead of including the image \emph{in} the document it will
link to the image, i.e.\ instead of,

\begin{verbatim}
<IMG SRC="overview-kernel.gif">
\end{verbatim}

we would use,

\begin{verbatim}
<A HREF="overview-kernel.gif" BPP=1 MAXWIDTH=700 MAXHEIGHT=700>overview-kernel.gif</A>
\end{verbatim}

Then we can tap on the link to the image when we want to view it.
This is something we can only do when we have access to the document
and to support this in a more transparent way the parser should be
able to do this automatically for you in the future.\\

\begin{latexonly}
Since we are not interested in any external documents we will use the
\longoption{Spider.py}{-{}-stayonhost} option and a high maximum depth.
Then we don't have to worry about exactly how deep we should follow
links and what external links we should filter out using exclusion lists.
Now we are ready to build the document,\\

\indata{\% Spider.py -v -{}-stayonhost -M5 -H file:/tmp/sag/index.html -N "Linux Admin Guide" -f DB/SAG}
\end{latexonly}
\begin{htmlonly}
Since we are not interested in any external documents we will use the
\longoption{Spider.py}{---stayonhost} option and a high maximum depth.
Then we don't have to worry about exactly how deep we should follow
links and what external links we should filter out using exclusion lists.
Now we are ready to build the document,

\indata{\% Spider.py -v ---stayonhost -M5 -H file:/tmp/sag/index.html -N "Linux Admin Guide" -f DB/SAG}
\end{htmlonly}\\

\begin{verbatim}
Working for pluckerdir /home/pilot/.plucker
Processing file:/tmp/sag/index.html.
           0 collected, 0 still to do
  Retrieved ok

                    :

Processing file:/tmp/sag/backup-timeline.gif.
           73 collected, 0 still to do
  Retrieved ok

Writing out collected data...
Writing db 'Linux Admin Guide' to file /home/pilot/.plucker/SAG.pdb
Converted file:/tmp/sag/book1.html

                    :

Converted file:/tmp/sag/x89.html
Wrote 1 <= plucker:/~special~/index
Wrote 2 <= file:/tmp/sag/index.html
Wrote 3 <= plucker:/~special~/pluckerlinks
Wrote 11 <= file:/tmp/sag/backup-timeline.gif

                    :

Wrote 83 <= mailto:gregh@sunsite.unc.edu
Wrote 87 <= plucker:/~special~/links1
Done!

\end{verbatim}

Install the document you find in \name{/home/pilot/.plucker/DB} and
you have instant access to \emph{The Linux System Administrators' Guide}.


\subsection{Pluck remote targets}

E-books and manuals in all glory, but many times we want to get fresh
articles that updates every day. Well, Plucker can handle that, too.

\subsubsection{Daily News}

To be able to get the latest news from \name{Wired} we will set up a
special section in the configuration file, so that we only have to run,\\

\indata{\% Spider.py -s wired}\\

every morning to get the latest version of \name{Wired} for later perusal.\\

The handheld friendly version of \name{Wired} is located at
\name{\htmladdnormallink{http://www.wired.com/news\_drop/palmpilot/}
{http://www.wired.com/news\_drop/palmpilot/}} and we want to pluck it
to a depth of 3 levels. We also know that it only uses black and white
images. This give us the following section in the configuration file
(\name{/home/pilot/.pluckerrc}),

\begin{verbatim}
[wired]
bpp = 1
home_maxdepth = 3
home_url = http://www.wired.com/news_drop/palmpilot/
db_file = DB/Wired
\end{verbatim}

\subsubsection{Comics}

We need some fun, too, so let's download a few strips for some well
known comics. To simplify things we will use a tool called \name{netcomics}
to get the comics and then use a local description file to build the
document. How to install \name{netcomics} is beyond this tutorial,
but it is a Perl script and might work on any platform that have Perl
support (for Linux users there exists pre-built packages). After you
have installed \name{netcomics}, you should create a small shellscript
called \name{netcomics.sh} to be used by the parser,

\begin{verbatim}
#!/bin/sh

netcomics -D -d /tmp/Comics/ -c "ch dilbert dilbertcl uf"

( cd /tmp/Comics ; \
mv Dilbert-*.gif Dilbert.gif ; \
mv Dilbert_Classics-*.gif Dilbert_Classics.gif ; \
mv Calvin_and_Hobbes-*.gif Calvin_and_Hobbes.gif ; \
mv User_Friendly-*.gif User_Friendly.gif )
\end{verbatim}

On OS/2 and Windows this will look like the follwing. On OS/2 it should
be named \code{netcomics.cmd} whereas on Windows it should be named 
\code{netcomics.bat}:

\begin{verbatim}
perl netcomics.pl -D -d \temp\Comics\ -c "ch dilbert dilbertcl uf"

cd \temp\Comics
move Dilbert-*.gif Dilbert.gif
move Dilbert_Classics-*.gif Dilbert_Classics.gif
move Calvin_and_Hobbes-*.gif Calvin_and_Hobbes.gif
move User_Friendly-*.gif User_Friendly.gif

\end{verbatim}

This script will download Calvin \& Hobbes, Dilbert, Dilbert Classic
and UserFriendly to a separate directory (\name{/tmp/Comics/}) and
rename the date specific files into a general format that can be used
in the local description file,

\begin{verbatim}
<HTML>
<BODY>

<H1>Comics Home Page</H1>

<A HREF="file:/tmp/Comics/Dilbert.gif">Dilbert</A><P>
<A HREF="file:/tmp/Comics/Dilbert_Classics.gif">Dilbert Classic</A><P>
<A HREF="file:/tmp/Comics/Calvin_and_Hobbes.gif">Calvin &amp; Hobbes</A><P>
<A HREF="file:/tmp/Comics/User_Friendly.gif">UserFriendly</A><P>

</BODY>
</HTML>
\end{verbatim}

To simplify things even further we will also add a new section for the
comics,

\begin{verbatim}
[comics]
bpp = 4
home_url = plucker:/HTML/comics.html
maxwidth = 600
maxheight = 200
db_file = DB/Comics
before_command = "netcomics.sh"
\end{verbatim}

\note On OS/2 or Windows you can use the before\_command to the set
the name of your batch file.

As you can see we have added the shellscript as a command that should
be run \emph{before} the description file is parsed. Everyday (except
on Sunday when the strips are too large for these options --- we will
show a solution to that later in the section) we now only have to run,\\

\indata{\% Spider.py -v -s comics}
\begin{verbatim}
Executing 'before_command': "netcomics.sh"
Working for pluckerdir /home/pilot/.plucker
Processing file:/home/pilot/.plucker/HTML/comics.html.
           0 collected, 0 still to do
  Retrieved ok
Processing file:/tmp/Comics/Dilbert.gif.
           1 collected, 3 still to do
  Retrieved ok
Processing file:/tmp/Comics/Dilbert_Classics.gif.
           2 collected, 2 still to do
  Retrieved ok
Processing file:/tmp/Comics/Calvin_and_Hobbes.gif.
           3 collected, 1 still to do
  Retrieved ok
Processing file:/tmp/Comics/User_Friendly.gif.
           4 collected, 0 still to do
  Retrieved ok

Writing out collected data...
Writing db 'Comics' to file /home/pilot/.plucker/DB/Comics.pdb
Converted file:/home/pilot/.plucker/HTML/comics.html
Wrote 1 <= plucker:/~special~/index
Wrote 2 <= file:/home/pilot/.plucker/HTML/comics.html
Wrote 3 <= plucker:/~special~/pluckerlinks
Wrote 11 <= file:/tmp/Comics/Calvin_and_Hobbes.gif
Wrote 12 <= file:/tmp/Comics/Dilbert.gif
Wrote 13 <= file:/tmp/Comics/Dilbert_Classics.gif
Wrote 14 <= file:/tmp/Comics/User_Friendly.gif
Wrote 15 <= plucker:/~special~/links1
Done!

\end{verbatim}

To be able to use it also on Sundays we add yet another section to
the configuration file,

\begin{verbatim}
[sunday]
bpp = 2
maxwidth = 550
maxheight = 400
db_file = DB/SundayComics
\end{verbatim}

Using a lower bit depth for the images we are now able to include
larger versions of the comics.  Each Sunday we would run,\\

\indata{\% Spider.py -s comics -s sunday}\\

and since the parser applies the sections in the given order the
changed values in \name{sunday} will override the ones in \name{comics}.

